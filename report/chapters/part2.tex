\chapter{Part 2}
\label{ch:into} % This how you label a chapter and the key (e.g., ch:into) will be used to refer this chapter ``Introduction'' later in the report. 
% the key ``ch:into'' can be used with command \ref{ch:intor} to refere this Chapter.

La seconda parte dell'assignement consisteva nell'implementazione di una versoine di Pixel art, un software per la colorazione 
collaborativa di una matrice, tramite un gestore di scambi di messaggi. Le modifiche sulla propria matrice e ricezione di quelle degli altri utenti 
devono essere visibili sulla propria "versione locale" senza dover condividere nessuna porzione di memoria.

Durante la creazione di ogni client viene attributo un clientID randomico. Ogni messaggio viene inviato sullo stesso \textbf{Exchange} che in questo contesto
è di tipo Topic, per poter inviare messaggi di determinati contesti in code separate. I contesti di interesse per questo esercizio, e i loro utilizzi sono i seguenti: 
\begin{itemize}
    \item \textbf{Mouse}: scambio della posizione dei puntatori di tutti i client collegati alla rete per poterli visualizzare contemporaneamente.
    \item \textbf{Brush}: notifica di cambio colore real-time del brush dei client.
    \item \textbf{Color}: evento di colorazione di un pixel dei client.
    \item \textbf{History}: ???? storia di eventi di colorazione pixel per i nuovi client che si collegano in un momento successivo. %%TODO check this
    \item \textbf{Exit}: ???? notifica di uscita di un client e rimozione del brush.%%TODO check this
\end{itemize}

La pubblicazione di messaggi in queste code è, giustamente, collegata agli eventi che vogliono catturare.
Prendendo ad esempio il cambio di colore di un brush. La classe di gestione dei brush è stata estesa aggiungendo a ciascun brush l'identificatore del proprio client.
Ogni messaggio sulla coda dei brushes, viene intercettato e poi passato a questa classe che contiene la lista di brush e riesce ad identificare il brush che ha cambiato colore
e lo mostra sul client locale che ha ricevuto il messaggio. Nello stesso modo l'evento di modifica di colore notifica questo evento pubblicando un messaggio che contiene le informazioni
necessarie agli altri client per modificare la grafica. visto la semplicità del contesto i messaggi contengono tutte le informazioni, come identificatore e colore, in una stringa 
separate da un carattere speciale, in questo caso il '_'. 

Un ragionamento analogo è stato fatto per ogni altro evento catturato utile al funzionamento corretto dell'applicazione. 